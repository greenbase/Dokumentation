\chapter{Vorausschauende Instandhaltung}
\label{ch:vorausschauende instandhaltung}
WAS IST PDM? VORFÜR IST ES GUT? WELCHEN BEZUG HAT ES ZUR AUFGABENSTELLUNG?

Kernelement der vorausschauenden Instandhaltung ist es eine bevorstehende Störung vorzeitig zu erkennen und entsprechende Maßnahmen zu ergreifen, um den Störungsfall zu verhindern. PDM ist also wie die präventive Instandhaltung darauf bedacht Störungen aktiv zu verhindern, löst Maßnahmen jedoch erst aus, wenn Hinweise auf eine baldige Störung vorliegen.

PDM wird häufig im Zusammenspiel mit Condition Monitoring umgesetzt. Letzteres kann entweder in Form von kontinuierlicher Überwachung von Anlangen mit Hilfe von Sensoren oder durch regelmäßige Inspektionen durchgeführt werden.

In jeden Fall kann durch die vorzeitige Feststellung von Hinweisen auf einen kritschen Schaden Maßnahmen zum optimalen Zeitpunkt durchgeführt werden. Einerseits um die Lebensdauer von Komponenten voll ausschöpfen zu können. Andererseits den Arbeitsplan möglichst effizient zu gestalten.

Die bedarfsgerechte Planbarkeit von Instandhaltungsmaßnahmen ermöglicht außerdem Ersatzteile und Arbeitsmaterial termingerecht zu beschaffen. Im Gegensatz zur präventiven Instandhaltung, bietet dies den Vorteil gesenker Lagerkosten. Bei PM kann Arbeitsmaterial zwar auch termingerecht zum Wartungsintervall bestellt werden, unerwartete Ausfälle lassen sich so jedoch nicht gänzlich vermeiden, was wiederrum Lagerhaltung notwendig macht. 

