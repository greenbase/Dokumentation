\chapter{Usecase}
\label{ch:usecase}
WIE SIEHT DER POTENTIELLE USECASE AUS? WARUM IST ER SINNVOLL? WIE KANN ER UMGESETZT WERDEN? WIE KÖNNEN ML-MODELE DABEI VERWENDET WERDEN?

Im Folgenden wird der Usecase beschrieben der mittels der zu entwickelnden Modelle behandelt werden können soll.

Die Türschließanlage eines Schienenfahrzeugs öffnet und schließt eine Zugtür über einen Bautenzug, der von einem pneumatischen Zylinder angetrieben wird. Die Anlage verfügt über mehrere Bauteile die starken Verschleiß aufgrund von Reibung unterworfen sind. Dazu gehören Rollenlager, Gleitschienen und Seilrollen. 

Condition Mointoring, und die daraus erwachsenden Möglichkeiten zur vorausschauenden Wartung, sind in Allgemeinen nur für besonders teure oder extrem sicherheitsrelevante Komponenten wirtschaftlich. Da die gesamte Türschließanlage keine kritische Komponenten des Zuges ist, ist auch die Fragestellung nach einer besonders kritischen Komponente in diesen Subsystem redundant. Ein effizienter PDM-Usecase an solch einer Schließanlage würde, die Komponenten in Betracht ziehen, die relativ zu anderen häufig ausfallen oder teuer sind und somit Ausfälle hohe Kosten nach sich ziehen würden. Da Umlenkrollen wie eingangs erwähnt zu den Komponenten gehören, die starkem Verschleiß ausgesetzt sind, kann angenommen werden, dass Umlenkrollen eine sinnvollerweise zu überwachende Komponenten sind. 

Ziel des PdM-Usecases ist es Beschädigungen an den Umlenkrollen erkennen zu können. Dies soll anhand von Sensordaten durch ein maschinelles Lernmodel möglichsein; also ohne Demontage der Komponenten. Zunächst soll lediglich eine Unterscheidung beider Fälle möglich sein. Eine Modell zur zukünftigen Zustandsveränderung und konkreter Ausfallvorhersage wird zunächst nicht berücksichtigt.

Ziel des fiktiven PDM-Usecases ist es einen Algorithmus zu verwenden, um anhand von Messwerten auf den Zustand der Umlenkrollen schließen zu können. Dabei wird von einer starken Beschädigung ausgegangen, die jedoch noch nicht so weit fortgeschritten ist, dass sie ihre Funktion nicht mehr erfüllen kann und in Konsequenz die Türschließanlage ausfällt. 

Die Daten, die durch das Condition Mointoring aufgenommen werden sollen, richten sich nach den mechanischen Eigenschaften der Umlenkrollen und dem Funktionsprinzip der Türschließanlage. Es ist zu erwarten, dass sich mit zunehmender Beschädigung der Umlenkrolle dessen Vibrationsprofil ändert. Daher werden die Beschleunigungen entlang der drei Raumrichtungen, sowie die Winkelgeschwindigkeiten gemessen. 
Da sich mit zunehmendem Verschleiß Reibungsverluste vergrößern, ist zu erwarten, dass die pneumatische Leistung, die nötig ist um die Anlage zu betreiben ebenfalls steigen wird. Bedeuten tut dies, dass der Zylinder eine größere Kraft aufbringen muss, sprich ein größerer Druck in Zylinder herschen muss. Es wäre daher sinnvoll auch den Luftdruck in Zylinder zu messen.

Zusätzlich wird die Zeit aufgenohmen, die benötigt wird, um eine Öffen/Schleißen-Zyklus zu beenden. Es wird erwartet, das sich die Zyklusdauer verlängert, wenn sich die Reibverluste aufgrund von Verschleiß ansteigen.

\section{Anwendungsszenario}
\label{sec:anwendungsszenario_usecase}
WIE KANN DER USECASE IN DER ANWENDUNG CHRONOLOGISCH ABLAUFEN?

Während der Öffnungs- und Schließvorgänge in Bahnhof werden ausgewählte Sensordaten der Tür aufgezeichnet. Nach Abfahrt des Zuges werden die Daten durch einen Algorithmus ausgewertet. Das Ergebnis sollte dabei idealerweise vorliegen ehe der nächste Bahnhof erreicht wird. Wird festegestellt, dass die Tür kurz vor einem Ausfall steht, kann so die Auswrikungen auf den Fahrgast minimiert werden. 

Angenommen es kann anhand der Daten und bekannter Verschleißzustände eine Progrnose über die verbleidende Lebensdauer abgegeben werden, kann der Austausch der Umlenkrollen geplant werden.

Es wird festegestellt, dass es innerhalb der nächsten zwei Wochen zu einem Bruch der Umlenkrollen kommen wird; vorrausgesetzt die Rollen wird weiter wie bisher belastet. Da es in den Zeitrahmen passt, wird der Austausch auf den nächsten Wartungsarbeitsgang gelegt. Die Zeitspanne bis dahin ist ausreichend, um die nötige ersetzenden Komponenten beim Zulieferer zu bestellen, sodass sie punktlich zum WArtungstermin bereit steht. Zusammen mit Routinearbeiten, wie der Reinigung der Zuges durch den Betreiber, werden auch die Umlenkrollen der Tür ausgewechselt. Durch den zeitnahen Austausch kurz vor den vorhergesagtem Ausfall können Ausfälle vermieden werden und gleichzeitig die Lebensdauer ideal ausgenutzt werden. 

\section{Einordnung in den Kontext der Bahntechnik}
\label{sec:kontext_bahntechnik_von_usecase}
WELCHE ANFORDUNGEN ERHEBT DIE PRAXIS AN DEN USECASE? BESTEHT BEDARF AM DEM USECASE?

Der beschriebene Usecase beruht auf der Annahme, dass die in \cref{sec:vorteile_nachteile_pdm} genannten Vorteile zutreffen. 

Der vorgeschlagene Usecase soll nun im Rahmen der praktischen Bahntechnik diskuitert werden; insbesondere in Hinblick darauf ob der Usecase einen Mehrwert gegenüber der aktuellen Instandhaltungsstratigie darstellt. Zu diesem Zweck werden die Aussagen von Sachverstängigen der Bahntechnik im Folgenden zusammengefasst.

\subsection{Interview: Hochbahn Hamburg}
\label{subsec:interview_hochbahn}
BESTEHT BEDARF AN DEM USECASE? WELCHE ANFORDERUNGEN WERDEN AN EINEN MÖGLICHEN USECASE GESTELLT? MACHT DIE EINFÜHRUNG DES USECASES SINN? WELCHE AUSWIRKUNGEN HABEN AUSFÄLLE? 

\textit{Interview mit Herr xxxxxxxxxxxxxxx am xx.xx.2020}

\subsection{Interview: Deutsche Bahn}
\label{subsec:interview_deutsche_bahn}
WIE LÄUFT DIE INSTANDHALTUNG AB? WIE WÜRDE DER USECASE DAS VORGEHEN BEI DER INSTANDHALTUNG VERÄNDERN? WAS MÜSSTE GESCHEHEN, UM DEN USECASE UMZUSETZEN? KANN DIE VERBESSERUNG GEGENÜBER DER AKTUELLEN INSTANDHALTUNGSSTRATEGIE QUANTIFIZIERT WERDEN?

\textit{Interview mit Herr xxxxxxxxxxxxxxx am xx.xx.2020}

\section{Erfolgskriterien}
\label{sec:erfolgskriterien_usecase}

Aufgrund der grundlegend unterschiedlichen Natur von PM und PdM, lassen sich nur wenige Vergleichende kriterien bestimmen. Um zu bestimmen welche Strategie besser für einen Usecase geeignet ist sollten, vor einer Umstellung ausgiebige Kostenabschätzungen durchgeführt werden. Ggf. können noch weitere Faktoren, wie die Möglichkeit der Prozesssteuerung, mit in eine Entscheidung einfließen.

Dennoch können Kriterien definiert werden, an denen der Nutzen des PdM-Usecases gemessen werden kann. 

\textbf{Komplexität}

Ein Kriterium ist die Komplexität des maschinellen Lernmodels. Idealerweise soll das Modell nicht nur möglichst konkrete Vorhersagen zur Zustandsänderung abgeben können, sondern auch Informationen extrahieren, die für Verbesserungen der Konstruktion verwendet werden können. Dafür ist ein Model nötig dessen Entscheidungverfahren nachvollziehbar ist. Ein zu komplexes Model kann nicht unter praktikablen Bedingungen ausgewertet werden. Aus diesem Grund ist ein einfach nachzuvollziehendes Model einem komplexeren vorzuziehen.

\textbf{Qualität der Vorhersagen}

Damit der Usecase einen Mehrwert gegenüber dem vorherigen Vorgehen bietet, müssen die Vorhersagen des Models ein Mindestmaß an Genauigkeit bieten können. Sowohl falsch positive; also Fehlalarme; wie auch falsch negative; unentdeckte Beschädigungen; Vorhersagen, sollten möglichst gering sein und dem jeweiligen Mindestmaß gerecht werden.
Dabei sind falsch positive Vorhersagen geringer zu wichten als falsch negative. Falsch positve Vorhersagen haben eine unnötige Inspektion oder Austausch der Komponente zufolge, wohingegen falsch negative Vorhersagen zu einem unerwarteten Ausfall führen. Im ersten Fall entstehen Kosten durch Verschwendung; im zweiten durch die notwendige Behebung der Störung. Letzteres ist jedoch in der Regel mit vergleichsweise höheren Kosten verbunden. Deswegen wird falsch negativen Vorhersagen ein höheres Gewicht zugeschrieben als falsch positiven Vorhersagen. 
Die Qualität der Vorhersagen ist einerseits wichtig für die Abschätzung entstehender kosten des gesamten Usecases. Das Kriterium lässt sich in Form der Sensitivität und Relevanz direkt auf die Bewertung des verwendeten Vorhersagemodels anwenden.

