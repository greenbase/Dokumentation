\chapter{Grundlagen der vorausschauenden Instandhaltung}
\label{ch:grundlagen_pdm}

Die Norm DIN 31051 definiert die Instandhaltung als "Kombination aller technischen und administrativen Maßnahmen sowie Maßnahmen des Managements während des \textit{Lebenszyklus […]} eines \textit{Objekts […]}, die dem Erhalt oder der Wiederherstellung ihres funktionsfähigen Zustands dient, sodass es die \textit{geforderte Funktion […]} erfüllen kann".~\cite[S.~4]{DIN.2019} Weiter werden Wartung, Inspektion, Instandsetzung und Verbesserungen als Kategorien der Instandhaltung genannt und definiert. 

Man unterscheidet zwischen verschiedenen Instandhaltung\textit{strategien}, die sich darin unterscheiden, wie die Maßnahmen im Rahmen der Instandhaltung konkret umgesetzt werden. Heute wird der Regel zwischen \textit{korriegierender}, \textit{präventiver} und \textit{vorausschauender} Instandhaltung unterschieden. Eine einheitliche Verwendung dieser Begriffe in der Literatur gibt es jedoch nicht.

Historisch betrachtet sind die drei  Instandhaltungsstrategien in der Reihenfolge aufgekommen, in der sie hier sie hier genannt sind. Dabei konnte der Trend beobachtet werden, dass zunehmend mehr Informationen für die Veranlassung und Durchführung von Instandhaltungmaßnahmen herangezogen wurde. Grund dafür liegt in der Bestrebung die Maßnahmen möglichst effizient zu gestalten, die Objektlebensdauer zu maximieren und die Produktqualität zu steigern. 

So wird bei der korriegierenden Instandhaltung erst dann eine Maßnahme veranlasst, wenn ein System oder eine Anlage ausgefallen; sprich ihre Funktion nicht mehr erfüllen kann. Es werden keine Informationen über den Zustand der Anlage berücksichtig, sondern nur das Symtom in Form des Ausfalls.

Bei der präventiven Instandhaltung werden Maßnahmen geplant, die verhindern sollen, dass es überhaupt zu einem Ausfall kommt. Mittels Abschätzungen und Erfahrungswerten können beispielsweise Wartungsintervalle definiert werden, die kürzer sind als die geschätze Zeit bis zum Ausfall. Es werden also abschätzbare Information, Erfahrungen und ggf. aktuelle Inspektionsergebnisse in die Entscheidung um nötige Maßnahmen mit eingeschlossen. Daraus ergibt sich ein Kostenvorteil für die präventive Instandhaltung gegenüber der korriegierenden. Laut Mobley fallen die Kosten der korriegierenden Instandhaltung dreimal höher aus als bei der präventiven Instandhaltung.Aufgrund der vergleichweise hohen Kosten, die beim Breakdown Maintenance anfallen, ist diese Wartungsstrategie dreimal teurer als die präventive Wartung.~\cite[S.~3]{Mobley.2002}

Kern der vorausschauenden Instandhaltung (engl. \textit{Predictive Maintenance,kurz PDM}) ist es einen bevorstehenden Ausfall vorzeitig zu erkennen und durch entsprechende Maßnahmen verhindern zu können. Im Gegensatz zur präventiven Instandhaltung werden jedoch Informationen über den tatsächlichen Zustand der Anlage berücksichtig. Durch regelmäßige -- ggf. sogar kontinuierliche -- Überwachung können Instandhaltungsmaßnahmen bedarfsgerecht geplant werden. U.a. können so die Lebendauern von Bauteilen optimal ausgenutzt und Kosten für Ersatzteillagerung minimiert werden, weil der tatsächliche Ausfallzeitpunkt abgeschätzt werden kann.

PDM hat in den vergangenen Jahren an Aufmerksamkeit gewonnen. Das ist vorallem auf die Fortschritte bei maschinellem Lernen und die steigende Verfügbarkeit von Rechenkapazitäten und Sensortechnick zurückzuführen. Nichts desto trotz stellt PDM heute eine Nischanwendungen dar. Lediglich in hoch sicherheitsrelevanten Bereichen, wie der Nukleartechnik, oder sehr kostenintensiven Anlagen sind PDM-Anwendungen bereits etabliert. 

PDM hat neben der Vermeidung von Ausfällen noch eine zweite Anwendungsmöglichkeit; die Qualtitätssicherung.