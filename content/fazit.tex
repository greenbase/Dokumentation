\chapter{Zusammenfassung und Ausblick}
\label{ch:fazit}
Für Seilrollen einer Türschließanlage wurde ein Anwendungsfall für eine vorausschauende Wartung erdacht. Dazu wurden Kriterien erarbeitet, die es ermöglichen den Wert des Anwendungsfalls zu quantifizieren. Für den theoretischen Anwendungsfall wurden außerdem eine Einordnung in den Kontext praktischer Instandhaltungverfahren in der Bahntechnik und eine Bewertung des Bedarfs vorgenommen.

Um in die Problemstellung einzuführen wurden wichtige Grundlagen der vorausschauenden Instandhaltung beleuchtet, indem die Instandhaltung als solches, sowie verschiedene Instandhaltungsstrategien dargelegt wurden. Insbesondere wurde das Verbesserungspotential dargestellt, welche PdM gegenüber den anderen Instandhaltungsstrategien bietet.

Desweiteren wurden drei maschinelle Lernmodelle beschriebenen, die zur Lösung der Problemstellung herangezogen wurden. Die Wahl dieser drei Modeltypen wurde in Hinblick auf den zugrunde liegenden Anwendungsfall begründet. 

Um die Modelle zu entwickeln wurden Messreihen aufgezeichnet, die das Vibrationsprofil der Türschließanlage beschreiben. Die Qualität der Modelle wurde durch Zusammenfassung der Rohdaten in einem beschreibenden Datensatz verbessert. Außerdem wurde festgestellt, dass das Merkmal \enquote{gx\_median} einen hohen Informationsgehalt bezüglich der Kategorien der Datenpunkte aufweist. Diese Erkenntnis ist voraussichtlich für eine detailliertere Analyse des mechanischen Systems Türschließanlage nützlich; ebenso wie für die Optimierung des Datensatzes.

Die trainierten Modelle wurden im Rahmen einer Parameterstudie optimiert und anschließend bewertet, sodass ein Random Forest als das Model bestimmt werden konnte, das den Anforderungen an den PdM-Usecase am besten gerecht wird. Dabei ist zu beachten, dass das Model bislang nur zwischen zwei klar definierten Zuständen der Seilrolle unterscheiden kann. In aktuellen Stadium kann das Model zwar für eine zustandorientierte Wartung der Seilrollen dienen; es ist aber noch nicht für eine vorausschauende Instandhaltung im eigentlichen Sinne fähig. Dazu ist das Model auf weitere Beschädigungszustände zu erweitern und deren zugehörigen Betriebszeiten zu berücksichtigten.
%===============================================================================