\chapter{Fazit und Ausblick}
\label{ch:fazit}
WAS WURDE GEMACHT? WAS SIND DIE ERGEBNISSE? WO BESTEHT VERBESERUNGSMÖGLICHKEIT? 
IST DER USECASE GRUNDSÄTZLICH UND MIT DEM BESTEN MODEL BRAUCHBAR?

Für die Türschließanlage eines Intercity-Zuges wurde eine möglicher Anwendung der vorausschauenden Wartung erläutert und diskutiert. Es wurde erötert unter welchen Voraussetzungen der Usecase wirtschaftlich und technisch eine sinnvolle Verbesserung gegenüber herkömmlichen Instandhaltungsstrategien darstellt. Um den Nutzen des Usecases diskutieren zu können, wurde die nötige Auswertesoftware, in Form von maschinellen Lernalgorithmen, erstellt. Der Usecase wird auf Modelle maschinellen Lernens gestützt, die die Möglichkeit bieten Messwerte einer kontinuierlichen Überwachung der fraglichen Komponente zu untersuchen und Einschätzungen über den Zustand der Komponente abzugeben. Erst durch diese Modelle ist die Umsetzung des Usecases praktikabel. 

Um das am besten geeignete Model zu finden, wurde zunächst eine Vorauswahl in Frage kommender Modeltypen getroffen. Anschließend wurden Modele vom Typ Entscheidungsbäume, Random Forest und GBT trainiert. Dabei wurden deren Parameter durch eine Parameterstudie bestimmt, um eine möglichst gute Vorhersagequalität der Modelle sicherzustellen. Bei der Entwicklung der Modelle wurden Kenntnisse über die mechanischen Eigenschaften des untersuchten Anlage, wie über die Eigenschaften der Modeltypen berücksichtigt.

Ziel der Entwicklung mehrer Modelle war es das Model zu bestimmen das am besten für die Anwendung des vorgeschlagenen Usecases geeignet ist; also den großten Nutzwert trägt. Dieser Nutzwert wurde anhand ausgewählten Kriterien bestimmt, die für den gegebenen Usecase relevant sind. Unter den getroffenen Annahmen hat das Model vom Typ Random Forest den größten Nutzwert erzielt. Mit dem besten Model konnt der Gesamtnutzen des PDM-Usecases unter wirtschaftlichen und technischen Gesichtspunkten diskutiert werden.

Für nachfolgende Arbeiten erscheint die Einbeziehung des pneumatischen Druckverlaufs während der Zyklen als weiterer poteniell informationsreiches Merkmal. Es wird vermutet, dass sich die Vorhersagequalität der Modelle dadurch weiter verbessern lassen, ohne das der Aufwand zum Trainieren der Modelle dadurch maßgeblich steigt. 

Um die Ergebnisse bezüglich des Nutzen des Usecases zu verifizieren, ist es sinnvoll die zu erwartenden Schadensfälle der Umlenkrollen umfassender zu untersuchen. Das umfasst sowohl eine Kontrollreihe zu dem hier beschriebenen Experiment, als auch die Untersuchung anderer Schadensausprägungen an den Umlenkrollen. Um eine konkrete Einschätzung der verbleibenden Dauer bis zum Ausfall der Komponente abgeben zu können ist außerdem die Betrachung verschiedener Schädigungsgrade des gleichen Schadensbild sinnvoll. 