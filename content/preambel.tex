%Dokumentklasse und Spracheinstellung
%\documentclass[12pt,oneside,paper=A4,DIV=15,BCOR=0mm,abstract=true,headsepline,headings=normal]{scrreprt}
\documentclass[12pt,twoside,paper=A4,DIV=15,BCOR=12mm,abstract=true,headsepline,headings=normal,parskip=half,ngerman]{scrreprt}
\usepackage{scrhack}
\usepackage{babel}
\usepackage[utf8]{inputenc}

%Schriftart
\usepackage{libertine}
\usepackage{libertinust1math}
\usepackage[T1]{fontenc}
\usepackage{csquotes} % for german quotation

%Mathe, Symbole, EInheitendarstellung, Chemie
\usepackage{amsmath}
\usepackage{amsxtra}
\usepackage{eurosym}
\usepackage{siunitx}  
\sisetup{locale=DE}
\usepackage[version=4]{mhchem}
%\usepackage{units}
%\usepackage{cancel}

\usepackage[auto]{microtype}
\clubpenalty = 10000
\widowpenalty = 10000
\displaywidowpenalty = 10000

%Einbindung von Bildern, Tabellen, pdf-Seiten, Quellcode
\usepackage{graphicx}
\usepackage{multirow,multicol,booktabs}
\usepackage{threeparttable}
\usepackage{longtable}
\usepackage{rotating}
\usepackage{ltablex}
\usepackage{subfig}
\captionsetup[subtable]{position=top}
\usepackage{pdfpages}
\usepackage{listings}
\graphicspath{{images/}}	% define where to look for images

%Farben

%Darstellung von URL
\usepackage{url}
\urlstyle{same}

%Fussnoten, auch für Tabellen
\usepackage{footnote}
\makesavenoteenv{tabular} 

%Pakete für Kontrolle und Review
\usepackage{todonotes}
\usepackage{blindtext}

%Darstellung der Literaturangaben
\usepackage[
backend=biber,
style=numeric,
citestyle=numeric-comp,
maxbibnames=2,
firstinits=true
]{biblatex}

\renewcommand*{\labelnamepunct}{\addcolon\addspace}

%Speicherort der Literaturangaben (*.bib Datei)
\bibliography{literature/literaturdatenbank}

%Fussnoten
%Markierung in der Fußnote selbst weder hochgestellt noch kleiner gesetzt
%\deffootnote{1em}{1em}{\thefootnotemark\ }
%linksbündige Fußnotenmarkierungen
\deffootnote{1.5em}{1em}{%
	\makebox[1.5em][l]{\thefootnotemark}%
}

%Fussnoten nicht umbrechen
\interfootnotelinepenalty=10000

%Gestaltung der Bildunterschrift und Tabellenüberschirften sowieTitelseitenangaben
\addtokomafont{caption}{\small}
\setkomafont{captionlabel}{\sffamily\bfseries}
\setkomafont{author}{\large}
\setkomafont{date}{\large}
\setkomafont{publishers}{\large}

\renewcaptionname{ngerman}{\figurename}{Abb.}
\renewcaptionname{ngerman}{\tablename}{Tab.} 

%Tabellenumgebungen mit Schriftgröße 10 und 7
\usepackage{tabularx}
\newcolumntype{g}{>{\columncolor{Gray}}c} %grau hinterlegte Spalte; zentriert
\newcolumntype{L}{>{\raggedright\arraybackslash}X}

\newenvironment{tabular10}{%
	\fontsize{10}{12}\selectfont\tabular
}{%
	\endtabular
}

\newenvironment{tabular7}{%
	\fontsize{7}{12}\selectfont\tabular
}{%
	\endtabular
}



%Verweise und Refernezen, pdf-Eisntellungen
%Angaben aktualisieren! (Für korrekte PDF-Meta-Daten)
\usepackage[
pdftitle={Ausarbeitung eines Use-Cases zur vorrausschauenden Wartung von Zugtüren},
pdfsubject={Masterthesis},
pdfauthor={Daniel Mansfeldt},
pdfkeywords={Predictive Maintenance, PdM, Train},  
%Links nicht einrahmen
hidelinks
]{hyperref}
\usepackage[german]{cleveref}