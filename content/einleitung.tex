\chapter{Einleitung}
\label{ch:einleitung}

\section{Motivation}
\label{sec:motivation}
Textbreite:\the\textwidth

Text sdfkljasdk dasfd dsakfe dskalfj dsaflke dsd dkslf ssfe dsfafeklflsdkf dsf lsf. Text sdfkljasdk dasfd dsakfe dskalfj dsaflke dsd dkslf ssfe dsfafeklflsdkf dsf lsf. Text sdfkljasdk dasfd dsakfe dskalfj dsaflke dsd dkslf ssfe dsfafeklflsdkf dsf lsf. Text sdfkljasdk dasfd dsakfe dskalfj dsaflke dsd dkslf ssfe dsfafeklflsdkf dsf lsf. Text sdfkljasdk dasfd dsakfe dskalfj dsaflke dsd dkslf ssfe dsfafeklflsdkf dsf lsf.

\begin{figure}[ht]
	\centering
	\includegraphics[scale=0.4]{FH_Kiel_Logo_deut_rgb.jpg}
	\caption{Logo}
	\label{fig:fhlogo}
\end{figure}

dskalfj dsaflke dsd dkslf ssfe e dsd dkslf ssfe dsfafeklflsdkf dsf lsf. \cite{gasparovic1969}
\section{Ziel der Arbeit}
\label{sec:ziel}
Fokus der Arbeit ist es zu bestimmen in wie weit der simulierte Schadensfall mit ML-Modellen vorhergesagt werden kann.

Davon ausgehend sollen darüberhinaus Parallelen zu reallen potentiellen Anwendungsfällen gezogen werden. 

\section{Gliederung und Hinweise}
\label{sec:gliederung}

\blindtext[2]

\blindtext[3]

\blindtext[5]

\blindtext[3]