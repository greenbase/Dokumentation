\chapter{Einleitung}
\label{ch:einleitung}

\section{Problemstellung}
\label{sec:motivation}
Die vorausschauende Instandhaltung (engl. \textit{Predictive Maintenance}; kurz \textit{PdM}) bietet für viele Instandhaltungsarbeiten großes Verbesserungspotential. Insbesondere im Zusammenhang mit der Industrie~{4.0} hat PdM in den letzten Jahren viel an Bedeutung gewonnen. PdM rückt für zunehmend mehr Anwendungen in den Fokus, ist aber auch noch heute in vergleichsweise wenigen Anwendungen zu finden. 

Ein Grund dafür warum PdM weniger verbreitet ist als z.B. präventive Methoden ist zum einen der hohe Investitionsaufwand und zum anderen die Tatsache, dass in vielen Bereichen mögliche Anwendungsfälle nicht bekannt sind. Zwar ist es denkbar, dass für den speziellen Fall eine andere Herangehensweise die Bessere ist, jedoch selbst um dies festhalten zu können, muss das Potential für die jeweilige Anwendung analysiert werden. 

Grundsätzlich bedarf es einer umfassenden Analyse eines neuen PdM-Usecases ehe dieser Umgesetzt werden sollte.

Die vorausschauende Instandhaltung bietet im Vergleich zur präventiven Instandhaltung das Potentiale der Senkung der Instandhaltungskosten und einen darüber hinausgehenden Informationsgewinn der auch für Prozesssteuerung und Verbesserungen genutzt werden kann. Dies ist aber nur unter bestimmten Voraussetzungen der Fall, welche sich dazu vom betrachteten Fall abhängen.

Um eine Einschätzung über das Potential eines bestimmten Anwendungsfall abgeben zu können, ist es notwendig diesen so weit zu erarbeiteten, dass ein Vergleich mit dem zu ersetzenden Verfahren sinnvoll durchgeführt werden kann.

Für die vorausschauende Wartung sind Vorhersagemodelle nötig. Deren Qualität ist einer der wichtigste Faktor für den Nutzen des gesamten Anwendungsfalls. Ist beispielsweise die Verhinderung von Ausfällen und eine besserte Ausnutzung der Bauteillebensdauer das Ziel, so muss das Model Vorhersagen treffen die zu einem gewissen Maß zutreffend sind, um einem anderen Verfahren gegenüber bevorzugt werden zu können.

Mit dem Aufschwung der Industrie 4.0 und der damit einhergehenden stetig größer werdenden Menge verfügbarer Daten, rückt PdM weiter in den Fokus. Schließlich bieten sich durch eine große Menge Daten neue Möglichkeiten. Nicht nur können bestehende Anwendungen mit größerer Präzision ausgeführt werden; es ergeben sich auch grundlegend neue Anwendungsmöglichkeiten.

PDM ist in vielen technischen Bereichen insbesondere deshalb interessant, weil  dadurch folgende Potentiale erschlossen werden können \cite[S.~31]{Holzel.2019}. Diese sind in Abgrenzung zur präventiven Instandhaltung zu sehen:
\begin{itemize}
	\item Bedarfsgerechte Ersatzteilbereitstellung
	\item Instandhaltungsmaßnahmen können zum optimalen Zeitpunkt durchgeführt werden
	\item Verlängerung der Anlagen- und Bauteillebensdauer
	\item Reduzierte Stillstandzeiten
	\item Reduzierte Instandhaltungskosten
	\item Verbesserung der Anlagenzuverlässigkeit
\end{itemize}

Auch für die Instandhaltung von Schienenfahrzeugen erscheinen PDM-Anwendungen vielversprechend. Neben den oben genannten Punkten erscheint durch die Verringerung von Ausfällen auch Imageschaden abwendbar. Dieser Entsteht, wenn  der Ausfall einer Anlage (z.B. Zug oder Tür) sich direkt auf den Kunden auswirkt; in diesem Fall der Fahrgast.

Ein für Fahrgäste zu beobachtender Fall sind Türstörungen. In der Regel äußern sich diese darin, dass die Zugtür nicht mehr entsprechend der Sicherheitsanforderungen nutzbar sind. In den meisten Störungsfällen genügt es die Tür außer Betrieb zusetzen und um so die Sperrung eines Waggons oder gar den Ausfall einer ganzen Fahrt zu verhindern. Dennoch ist der Ausfall einer Zugtür für den Kunden direkt ersichtlich und führt damit zu einem Imageschaden.

Die Fragestellung, die bei der Umsetzung aller PDM-Anwendungen zunächst geklärt werden muss ist, ob das entwickelte Modell zur Vorhersage ausreichend präzise Vorhersagen trifft. Insbesondere entscheidend ist dabei der vergleich mit der gegenwärtigen Instandhaltungsstrategie.

In jedem Fall muss die Frage geklärt werden, ob sich die Umstellung auf ein vorrauschauendes Instandhaltungsprogramm lohnt. Ein umfangreiche Analyse der zu erwartenden Kosten stellt die Grundlage zur Beantwortung dieser Frage dar. Das Vorhersagemodel, auf dem der Anwendungsfall basiert, stellt die wichtigste Grundlage zur Kostenanalyse dar. Ohne eine zuverlässige Einschätzung über die Treffsicherheit der Vorhersagen, kann kein Kostenvergleich zu einem anderen Verfahren erstellt werden. Die Entwicklung eines Vorhersagemodels stellt deswegen eine entscheidende Grundlage dar. 
%===============================================================================
\section{Ziel der Arbeit}
\label{sec:ziel}
Motivation und zielführende Idee dieser Arbeit ist es für die Türschließanlage eines Intercity-Zuges der Deutschen Bahn einen PdM-Usecase zu erarbeiteten. Konkret soll die Wartung von Seilrollen, die Teil des Systems Türschließanlage sind, durch prädiktive Instandhaltung möglich sein.

Ziel dieser Arbeit ist es fundamentale Erkenntnisse zur erarbeiteten, auf denen Weiterentwicklungen hin zu einer praktikablen PdM-Anwendung aufbauen können. Dies umfasst zum einen eine Beschreibung und Diskussion des potentiellen Anwendungsfalls und zum anderen die Erstellung eines maschinellen Lernmodels. Das Model ist in der Lage zuverlässig zwischen zwei Beschädigungszuständen der Seilrollen zu unterscheiden. Die Modelentwicklung wird zunächst also auf eine Zustandserkennung; nicht auf Vorhersagen; ausgelegt.

Abschließend wird ein Ausblick gegeben, der mögliche nachfolgende Schritte beleuchtet, um den Anwendungsfall praxistauglich zu gestalten.
%===============================================================================
\section{Aufbau dieser Arbeit}
\label{sec:aufbau_dieser_arbeit}
Von \cref{ch:instandhaltungsstrategien} bis \cref{ch:machine_learning} wird der Stand der Technik der Elemente beschrieben, die für den Inhalt der Arbeit von Bedeutung sind.

\cref{ch:instandhaltungsstrategien} befasst sich mit verschiedenen Instandhaltungsstrategien, stellt Unterschiede heraus und diskutiert Potentiale und Voraussetzungen für prädiktive Instandhaltung.

In \cref{ch:machine_learning} werden Kernpunkte von maschinellem Lernen erörtert; ehe auf die Grundlagen der Lernmodelle eingegangen wird, die im weiteren Verlauf dieser Arbeit verwendet werden. Für die einzelnen Modelle werden u.a. Funktionsweise, Parametereinstellungen, Vor- und Nachteile, sowie die Eignung für den in \cref{ch:usecase} dargestellten Anwendungsfall besprochen. 

\cref{ch:usecase} stellt den neu erdachten PdM-Usecase vor. Um den Usecase in den Kontext der praktischen Bahntechnik zu setzen, werden die Inhalte zweier Interviews mit Sachverständigen der Bahntechnik rekapituliert. In Zuge dessen wird auch der reale Bedarf des vorgeschlagenen Usecases diskutiert. Die Darstellung des Usecases umfasst außerdem eine Zusammenstellung von Erfolgskriterien.

In \cref{ch:methodik} wird die Methodik beschrieben mit der der Usecase exemplarisch untersucht wurde. Es wird der Versuchaufbau beschrieben, eine Einschätzung über den Datensatz abgegeben und eine begründete Vorauswahl maschineller Lernmodelle getroffen. In \cref{sec:bewertungskriterien} werden die Erfolgskriterien des Usecases aus \cref{ch:usecase} aufgegriffen, um die zu erstellenden Modelle sinnvoll bewerten zu können. In \cref{sec:bewertungskriterien} wird mit diesen eine Präferenzfunktion zur Beurteilung der Modelle erstellt. Das Kapitel schließt mit einer Beschreibung des Vorgehens bei der Erstellung der Modelle.

In \cref{ch:modelbewertung} wird die Eignung der trainierten Modelle für den Usecase diskutiert. Es wird eine Einschätzung über die Qualität der Modelle abgegeben und anhand der Präferenzfunktion das Model bestimmt welches am besten für den Usecase geeignet ist.

\cref{ch:fazit} fasst die Ergebnisse der Arbeit zusammen und gibt einen Ausblick über mögliche Schritte zur Weiterentwicklungen des Usecases.
%===============================================================================