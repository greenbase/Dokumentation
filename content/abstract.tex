%------------------Deutsch------------------
% =>
\begin{center}
	\textbf{Kurzfassung}
\end{center}
\raggedright
Die vorausschauende Instandhaltung verspricht deutliche Verbesserungen bei der Wartung, Inspektion, Instandsetzung und Verbesserung von Systemen und Anlagen gegenüber der heute zu tage üblichen präventiven Instandhaltung. Insbesondere im Kontext der Industrie~\num{4.0} sind die Erwartungen an die Auswertung großer Datenmengen groß.

Trotz der großen Potentiale der vorausschauenden Instandhaltung sind bis lang relativ wenige Anwendungsfälle erschlossen. In der Regel ist eine präzise Abschätzung nötig ob sich die Implementierung eines PDM-Usecases lohnt.

Wenn auch nicht zwingend erforderlich, so kommen doch häufig maschinelle Lernalgorthmen für PDM-Usecases zum Einsatz. Diese sind für eine sinnvolle Umsetzung sorgfälltig auszuwählen und an die gegebenen Umstände anzupassen.

In dieser Arbeit soll ein potentieller PDM-Usecase für die Instandhaltung einer Zugtür erdacht werden; genauer für die Wartung der Umlenkrollen eines Bautenzuges. Es wird begründet welche Messwerte benötigt werden, um den Zustand der Umlenkrollen zu bestimmen, um daraus Maßnahmen für die Instandhaltung ableiten zu können. Mit den Daten wird anschließend eine Auswahl maschineller Lernmodel trainiert und das am besten für den Usecase geeignete Model bestimmt. Dazu wird sowohl auf die Eigenschaften der Modelle eingegangen, wie auch die Erfolgskriterien des Usecases berücksichtigt und die Ergebnisse in diesem Kontext diskutiert. Schlussendlich kann so eine Einschätzung abgegeben werden, ob der vorgeschlagene Usecase mit dem entwickelten Model eine brauchbare Anwendung von PDM darstellt.

Der Usecase wird außerdem im Kontext der aktuellen Bahntechnik diskuiert.
% <=
%-------------------------------------------
\clearpage
%------------------English------------------
% =>
\begin{center}
	\textbf{Abstract}
\end{center}
\raggedright
english
% <=
%-------------------------------------------
\clearpage