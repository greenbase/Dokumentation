%------------------Deutsch------------------
% =>
\begin{center}
	\textbf{Kurzfassung}
\end{center}
\raggedright
Wie in vielen technischen Bereichen stellt sich auch im Schienverkehr die Frage in welchen Fällen vorausschauende Instandhaltung etablierten Instandhaltungsstrategien überlegen ist und ob sich Vorteile daraus ziehen lassen.

Für den möglichen Fall einer beschädigten, aber noch funktionstüchtiger Umlenkrolle soll maschienellen Lernmodel entwickelt werden. Dieses soll in der Lage sein anhand von Messwerten bestimmen zu können, ob die Umlenkrolle intakt ist oder bereits beschädigt ist. So soll in der Konsequenz ein Ausfall der Komponente und damit der gesamten Anlage vermieden werden können.

Ziel dieser Arbeit ist in erster Linie die Qualität des Models zu bewertet werden; in Hinblick auf dessen Nutzen zur vorausschauenden Instandhaltung der Umlenkrolle. Dazu werden verschiedene Modelle anhand unterschiedlicher Kriterien beurteilt, die alle deren Kosten-Nutzen-Verhältniss beeinflussen. 
% <=
%-------------------------------------------
\clearpage
%------------------English------------------
% =>
\begin{center}
	\textbf{Abstract}
\end{center}
\raggedright
english
% <=
%-------------------------------------------
\clearpage