%------------------Deutsch------------------
% =>
\begin{center}
	\textbf{Kurzfassung}
\end{center}
\raggedright



Ziel dieser Arbeit ist es maschienelle Lernmodelle zu entwickeln und zu bewerten welches am besten für die Anwendung in einem PDM-Usecase auf dem simulierten Schadensfall geeignet ist. Die Modelle werden dazu hand verschiedener Kriterien beurteilt, die für die Wirtschaftlichkeit eines PDM-Usecases relevant sind. 

Als zu lösendes Problem wird ein Instandhaltungsproblem an einer türschließanlage herangezogen. Genauer wird der Zustand einer Umlenkrolle beobachtet. Ziel der Modelle ist es zuverlässig zwischen intakten und beschädigten Umlenkrollen unterscheiden zu können.
% <=
%-------------------------------------------
\clearpage
%------------------English------------------
% =>
\begin{center}
	\textbf{Abstract}
\end{center}
\raggedright
english
% <=
%-------------------------------------------
\clearpage