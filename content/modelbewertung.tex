\chapter{Modelbewertung und Diskussion}
\label{ch:modelbewertung}
Tabelle \todo{Referenz auf Tabelle einfügen} zeigt die Parameter der besten Modelle jedes Typs und deren Sensitivität und Relevanz. Zusätzlich ist der Gesamtscore aufgeführt der sich aus der Präferenzfunktion ergibt.

\todo{Tabelle einfügen}
\begin{tabular}{g|c|c|c|}
    \hline
    \rowcolor{Gray}
    Metrik & Decision Tree & Random Forest & Boosted Trees\\
    \hline
    Tiefe (max) & 1 & 1 & 1\\
    \hline
    Anzahl & 1 & 19 & 95\\
    Sensitivität & 0,87 & 0,96 & 1,00\\
    \hline
    Relevanz & 0,91 & 0,85 & 0,96\\
    \hline
    \hline
    Nutzwert & 0,77 & 1 & 0,94\\
\end{tabular}

Den Ergebnissen nach ist das Random Forest Modell am besten geeignet, um für Vorhersagen zu dem Usecase verwendet zu werden. 

Zwar ist sind Sensitivität und Relevanz des Random Forests um \num{0,04} bzw. \num{0,11} kleiner als beim zweit bestplatziertem GBT-Model, da für ist es mit 19 Bäumen jedoch deutlich weniger komplex.

Der einzelne Entscheidungsbaum erzielt zwar den gleichen Grad an Komplexität wie der Random Forest, unterliegt aber insgesamt bei der Qualität der Vorhersagen.

Auffällig ist, dass alle Modelle eine Baumtiefe von \num{1} aufweisen. Für den einfachen Entscheidungsbaum bedeutet es, dass nur ein Merkmal des Datensatzes für die Vorhersagen herangezogen wurde; der Median der Rotationsgeschwindigkeit um die X-Achse. 

Random Forest und GBT nutzen mehrere Merkmale. Jeder einzelnen Baum jedoch nur eins. Da der einzelne Entscheidungsbaum vergleich auch relativ gute Vorhersagen liefert, wird vermutet, dass "gx_median" eine sehr hohe Bedeutung gegenüber allen anderen Merkmalen hat. Alle drei Modelle nutzen dieses Merkmal.



welche Features verwendet welches Model